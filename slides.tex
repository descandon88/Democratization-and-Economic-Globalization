\documentclass{beamer}
%Information to be included in the title page:
\title{Democratization and Economic Globalization}
\author{Helen V. Milner and Bumba Mukherjee}
\institute{Udelar}
\date{2022}

\begin{document}

\frame{\titlepage}
% Introducción
\begin{frame}
\frametitle{Introducción - “Dos Tendencias Centrales”:}
\begin{itemize}

\item La primera es la “tercera ola de democratización”, que comenzó lentamente a fines de la década de 1970 con las transiciones en España y Portugal.
\item Se extendió a muchos países latinoamericanos en la década de 1980 y luego cobró impulso en la década de 1990, con Europa del Este, el ex Unión Soviética y partes de África y Asia. 
\item La democracia se ha convertido así en una tendencia mundial, y los países de todo el mundo se han vuelto más democráticos.
\end{itemize}

\end{frame}

% Introduccion

\begin{frame}
\frametitle{Introducción - “Dos Tendencias Centrales”:}
\begin{itemize}
\item La segunda se ha caracterizado por el intercambio cada vez más intenso de personas, bienes, información y dinero a través de las fronteras nacionales.
\item Conocido  como "globalización económica".
\item“la era de la globalización económica” (Friedman 1999).
\end{itemize}

\end{frame}

% Introduccion

\begin{frame}
\frametitle{Introducción - “Definición”:}
\begin{itemize}
\item “Globalización económica” se utiliza para describir fenómenos tan diversos como:
    \begin{itemize} 
    \item La liberalización comercial y financiera
    \item Los flujos de inmigración
    \item La globalización cultural y la revolución en la tecnología de la información
    \end{itemize}
\item Los académicos a menudo se centran en dos aspectos de la globalización: 
\begin{enumerate}
    \item la adopción de políticas de libre comercio “liberalización del comercio”.
    \item el avance hacia una mayor apertura financiera, es decir, la liberalización de las cuentas de capital.

\end{enumerate}
\end{itemize}

\end{frame}

% Preguntas a plantearse

\begin{frame}
\frametitle{Se plantean las siguientes preguntas:}
\begin{itemize}
\item ¿Qué aprendemos de la literatura sobre el efecto de las transiciones democráticas en la economía internacional, es decir, el comercio y la cuenta de capital?
\item ¿Tiene la democratización un impacto positivo en la liberalización de la cuenta de capital y la apertura comercial?
\item Existe evidencia empírica de una relación endógena, es decir, causalidad mutua, entre democracia y liberalización económica? 
\item ¿La liberalización económica influye positivamente en las transiciones democráticas y en el nivel de democracia en general?
\item ¿Por qué la democratización afecta la liberalización comercial y financiera en los países en desarrollo?
\item ¿Por qué y cómo conduce el proceso de democratización a la apertura comercial y financiera?
\end{itemize}
\end{frame}


\begin{frame}
\frametitle{Estructura del artículo}
Se responden estas preguntas en dos principales partes:
\begin{itemize}

\item Primera parte se discute la literatura empírica y teórica sobre impacto causal de la democracia en la apertura comercial y ejercicios empíricos para evaluar la relación entre democracia y apertura comercial.
\item En la segunda parte se revisa críticamente la literatura que aborda el efecto de la democracia en la liberalización de la cuentas de capital. 
Análisis de algunos estudios que sugieren que una mayor apertura de las cuentas de capital aumenta la probabilidad y el nivel de democracia. 

\end{itemize}
Las pruebas empíricas para evaluar la relación entre democracia y aperturas comerciales y cuentas de capital se las realiza con:
\begin{itemize}

\item Muestra: 130 países en desarrollo 
\item Periodo: 1975 - 2002.
\end{itemize}

Consideraciones: Se enfocan principalmente en los países en desarrollo porque son los que se han democratizado recientemente y han hecho los cambios más grandes a sus políticas económicas.

\end{frame}

\begin{frame}
\frametitle{Conclusiones}
La rápida proliferación de regímenes democráticos en los países en desarrollo en los últimos 30 años ha producido un animado debate sobre su relación con la globalización económica.

La investigación reciente en ciencias políticas se ha visto particularmente impulsada por dos preguntas clave sobre la política de la globalización: ¿la democratización provoca la liberalización del comercio y de la cuenta de capital? ¿La mayor apertura comercial y financiera aumenta la probabilidad de democratización en los países en desarrollo?

\end{frame}

\end{document}
