\documentclass{beamer}
%Information to be included in the title page:
\title{Democratization and Economic Globalization}
\author{Helen V. Milner and Bumba Mukherjee}
\institute{Udelar}
\date{2021}

\begin{document}

\frame{\titlepage}

\begin{frame}
\frametitle{Dos tendencias:}
\begin{enumerate}

\item La primera es la “tercera ola de democratización”, que comenzó lentamente a fines de la década de 1970 con las transiciones en España y Portugal, se extendió a muchos países latinoamericanos en la década de 1980 y luego cobró impulso en la década de 1990, sumergiendo a Europa del Este, el ex Unión Soviética y partes de África y Asia. La democracia se ha convertido así en una tendencia mundial, y los países de todo el mundo se han vuelto más democráticos.
\item La segunda tendencia central se ha caracterizado por el intercambio cada vez más intenso de personas, bienes, información y dinero a través de las fronteras nacionales. Este fenómeno se conoce comúnmente como globalización económica, y varios comentaristas han bautizado el período contemporáneo como “la era de la globalización económica” (Friedman 1999).
\end{enumerate}

\end{frame}

\begin{frame}
\frametitle{Se plantean las siguientes preguntas:}
\begin{itemize}
\item ¿Qué aprendemos de la literatura sobre el efecto de las transiciones democráticas en la economía internacional, es decir, el comercio y la cuenta de capital, la liberalización? ¿ La democratización un efecto positivo en la liberalización comercial y, por lo tanto, en la apertura comercial? ¿Tiene la democratización un impacto positivo en la liberalización de la cuenta de capital?
\item Existe evidencia empírica de una relación endógena, es decir, causalidad mutua, entre democracia y liberalización económica? ¿La liberalización económica influye positivamente en las transiciones democráticas y en el nivel de democracia en general?
\item ¿Por qué la democratización afecta la liberalización comercial y financiera en los países en desarrollo? ¿Por qué y cómo conduce el proceso de democratización a la apertura comercial y financiera?
\end{itemize}
\end{frame}

\begin{frame}
\frametitle{Sample frame title}
Se responden estas preguntas revisando sistemáticamente la literatura sobre la relación entre democracia y globalización económica en dos partes principales.
\end{frame}

\end{document}
